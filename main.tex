\documentclass[12pt]{article}
\usepackage[utf8]{inputenc}

\usepackage[margin=1in]{geometry}
\usepackage{lipsum}

\usepackage[backend=biber,style=ieee]{biblatex}
\addbibresource{sources.bib}

\usepackage{titling}
\newcommand{\subtitle}[1]{%
	\posttitle{%
		\par\end{center}
	\begin{center}\large#1\end{center}
	\vskip0.1em}}%

\title{Analytic Research Memo 2\\
Analyze an Act}
\subtitle{PEGN 430A}
\author{Tyler Singleton}
\date{24 February 2022}

\begin{document}
\maketitle

\newpage
\setlength{\parindent}{0pt}

% --- Questions Section --- %
\textbf{Questions} \\

% Question 1
\textbf{1. Provide the full name of the Act and provide the legal citation.} \\

Act: 
Marine Mammal Protection Act of 1972 
\\

Legal Citation:
16 U.S.C \textsection 1361 \textit{et seq}.
\\

% Question 2
\textbf{2. What is the purpose of the Act?} \\

From \textsection 1361 Purposes and Findings, there are three main purposes of this act. Firstly to satisfy the International Dolphin Conservation Program, which is a multilateral agreement in establishing protections for dolphins and other marine mammals, and proliferate conservation and management efforts toward tuna inhabiting the eastern tropical Pacific Ocean. Secondly, this act establishes recognition for nation fisheries that have promoted ecologically stable fishing practices to reduce dolphin mortality. And thirdly, to allow tuna imports from nations practicing full compliance with the International Dolphin Conservation Program.
 \\

% Question 3
\textbf{3. Why is this Act important?} \\

In \textsection 1361 Congressional findings and declarations of policy, congress lay out the basis for why the Marine Mammal Protection Act (MMPA) is important. The general consensus is denoted by concern for extinction or significant depletion of marine mammals due to human activity. Ecosystems are complex and we lack adequate knowledge of marine mammal ecology and population dynamics, so we should be considerate in our takings of marine mammals, marine mammal products, and habitat. Because of this, it is important to promote protection and conservation of marine mammals and their habitats so that there remains a balance which allows for future generations to enjoy availability of marine mammals and their products. Congress continues to elaborate this act to be important as marine mammals are significant to international aesthetic, recreational, and economic commerce.
\\

% Question 4
\textbf{4. Research a case or example of a violation of the Act that you selected. Identify the following:} \\

\textbf{a) Parties to the case or issue} \\

United States v. David Hayashi \\

\textbf{b) facts of the case or issue} \\

The facts of this case revolve around a David Hayashi, a non-commercial fisherman, firing two rifle rounds at a group of porpoises to dissuade them from eating the tuna which he caught \cite{PaceLawReview}. \\

\textbf{c) provisions of the Act that are at issue or allegedly violated}\\

Hayashi was convicted of violating the ``Takings'' of the MMPA \textsection 1362(12) which defined an attempt to harass any marine mammal as a take \cite{PaceLawReview} \\

\textbf{d) procedural history of the case if it went to court or development of the issue from start to present time} \\

The procedural history of this case begins with the United States Attorney's Office convicting Hayashi before a Magistrate Judge, for intentionally taking (harassing) marine mammals in violation of the MMPA \cite{PaceLawReview}. Hayashi submits a request for appeal to the District Court of Hawaii, which was denied \cite{PaceLawReview}. Hayashi then requests for appeal to the United States Circuit Court of Appeals, Ninth Circuit which overturned his conviction \cite{PaceLawReview}. \\

\textbf{e) current status of the matter and the resolution if it there is one} \\

The Ninth Circuit later overturned his conviction because the MMPA did not have any regulation in regards to reasonably deterring marine mammals from eating a fisherman's catch \cite{PaceLawReview}. Due to this case and the court's decision, congresses amended the MMPA to allow for general deterrence of marine mammals fishing gear, catches, and public/private property \cite{PaceLawReview}.\\

\textbf{f) Do you agree with how the matter is presently being handled or the resolution if it has already occurred?  Provide support for your answer.} \\

I would agree with how the convection was overturned. Creating a precedent where fishermen cannot protect their catches and/or equipment from marine mammals without criminal convection would deter non-commercial or part-time commercial fishermen from competing with commercial fisheries. This would be due to commercial fisheries having the capital and resources to implement more ecologically friendly means of deterrents inline with the MMPA. However for a non-commercial fisherman, a rifle is a cheap solution and effective solution. \\

% --- Reflection Questions --- %
\textbf{Reflection Questions} \\

%RQ1
\textbf{RQ1. Why did you choose this Act?? } \\

I choose this act because of the amount of online resources available surrounding this act. NOAA has a page dedicated to the MMPA, numerous court cases, and collection of environmental law reviews for significant court cases. \\

%RQ2
\textbf{RQ2. What resources did you use while working on this assignment? Which ones were especially helpful?  Which ones would you use again?} \\

I mostly used the act itself. The introductory sections which listed the purposes and background of the act were extremely helpful in answering questions for this assignment. Also the United States Department of Justice's website was helpful in finding court cases, so I would use this website again. \\

%RQ3
\textbf{RQ3. What was especially satisfying to you about either the process or finished product?} \\

There was not anything especially satisfying to me for the process or product. Becoming more proficient with legal citation, and in some of the law reviews, observing how they cite the relevant sections, definitions, and court cases was helpful. \\

%RQ4
\textbf{RQ4. What was challenging for you in this assignment, and how did you overcome this challenge?} \\

This most challenging part of this assignment was finding an interesting court case. So looking at the amendments and searching for court cases around the same year was how I could overcome this. Court cases that cause changes substantial changes to laws are the ones which interest me the most. \\

%RQ5
\textbf{RQ5. What did you learn about yourself as you worked on this first ARM assignment?} \\

I learned that using it is not worth the time to research laws which have a small online profile. Even if I think the law is interesting, it is not an efficient use of time. Also that Microsoft Word gets more annoying to use each year with its automatic formatting. \\

%RQ6
\textbf{RQ6. What goals will you set for yourself to achieve success for the next individual assignments?} \\

My goals for the next assignments will mostly focus on becoming more proficient with legal citations. That is what I will struggle on the most within this course. 
\\

\newpage
\printbibliography


\end{document}
